%!TEX root = ../thesis.tex
\chapter{盾构隧道服役性能评估方法}

在道路工程领域,同样有对结构服役性能评估的需求,服役性能的影响因素很多,从机理方面研究的出一个综合性的服役性能指标是件困难的事,以往的研究经验表明(Saito和Sinha,\citeyear{saito1991delphi};Kushida等,\citeyear{kushida1997development}),专家打分在实际工程中是可操作性强、效率高的方法。

早在二十世纪六十年代,美国国家公路协会(American Association of State Highway Officials,AASHO)选定了伊利诺伊州、印第安纳州和明尼苏达州的72个沥青混凝土公路段和54个水泥混凝土公路段,组织行业专家对所选的路段进行乘车体验和路面观察,之后根据体验和观察结果对公路对进行服役性能打分,评估等级取值为1-5,其中1表示很差,2表示差,3表示一般,4表示好,5表示很好。与此同时,测量和收集与公路服役性能相关的指标,包括路面平整度、

%%%%%%%%%%%%%%%%%%%%%%%%%%%%%%%%%%%%%%%%%%%%%%%%%%%%%%%%%%%%%%%%%%%
\section{盾构隧道服役性能定义与假设}

正文




%%%%%%%%%%%%%%%%%%%%%%%%%%%%%%%%%%%%%%%%%%%%%%%%%%%%%%%%%%%%%%%%%%%
\section{数据采集与处理}

正文





%%%%%%%%%%%%%%%%%%%%%%%%%%%%%%%%%%%%%%%%%%%%%%%%%%%%%%%%%%%%%%%%%%%
\section{盾构隧道服役性能评估指标}

正文




%+++++++++++++++++++++++++++++++++++++++++++++++++++++++++++++++++%
\subsection{沉降}





%+++++++++++++++++++++++++++++++++++++++++++++++++++++++++++++++++%
\subsection{收敛}





%+++++++++++++++++++++++++++++++++++++++++++++++++++++++++++++++++%
\subsection{病害}






%%%%%%%%%%%%%%%%%%%%%%%%%%%%%%%%%%%%%%%%%%%%%%%%%%%%%%%%%%%%%%%%%%%
\section{盾构隧道服役性能评估方法}

正文





%%%%%%%%%%%%%%%%%%%%%%%%%%%%%%%%%%%%%%%%%%%%%%%%%%%%%%%%%%%%%%%%%%%
\section{本章小结}

正文