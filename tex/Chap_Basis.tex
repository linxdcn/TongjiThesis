%!TEX root = ../thesis.tex
\chapter{理论基础}
\label{chap:basis}

\section{引言}

铺垫几句……

\section{XX理论}

某年某大神发现……

\subsection{理论分支一}

先看一个我们都见过的公式
\begin{equation}
	E = m c^2
\end{equation}
其中$E$表示……

\subsection{理论分支二}

再看一些图表。图\ref{fig:logo}是我们的校徽和校名,也出现在了本论文模板的封面上。不用在意表\ref{tab:events}展示的内容,它只是个示例。
\begin{figure}[!h]
	\centering
	\includegraphics[width=0.75\textwidth]{tongji-whole-logo.eps}
	\caption{校徽和校名}
	\label{fig:logo}
\end{figure}

效果二

\begin{figure}[!h] 
	\centering 
	\begin{tabular}{cc} 
		\includegraphics[width=6cm]{tongji-whole-logo.eps} & \includegraphics[width=6cm]{tongji-whole-logo.eps}\\ 
		(a)图1 & (b)图2
	\end{tabular} 
	\caption{同济大学} 
	\label{fig:subimage} 
\end{figure}

效果三

\begin{figure}[!h] 
	\noindent 
	\begin{minipage}[t]{.48\linewidth} 
		\centering 
			\includegraphics[width=6cm]{tongji-whole-logo.eps} 
		\caption{图1} 
		\label{fig:quadratic} 
	\end{minipage} 
	\begin{minipage}[t]{.48\linewidth} 
		\centering 
		\includegraphics[width=6cm]{tongji-whole-logo.eps} 
			\caption{图2} 
		\label{fig:triorder} 
	\end{minipage} 
\end{figure}

表格效果

\begin{table}[!htb]
  	\centering
  	\caption{Add caption}
  	\label{tab:events}
    \begin{tabular}{?c"c?}
	    \thickhline
	    a     & b \bigstrut\\
	    \thinhline
		第一列   & 第二列 \bigstrut\\
	    \thinhline
	    第一列   & 第二列 \bigstrut\\
	    \thinhline
	    第一列   & 第二列 \bigstrut\\
	    \thickhline
    \end{tabular}%
  \label{tab:addlabel}
\end{table}



\subsection{本章小结}
本章……