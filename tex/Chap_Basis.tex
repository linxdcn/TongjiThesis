%!TEX root = ../thesis.tex
\chapter{理论基础}
\label{chap:basis}

\section{引言}

该模板由linxdcn:https://github.com/linxdcn/TongjiThesis~修改维护。

模板最初由Tongji LUG创作,2014年,https://sourceforge.net/projects/tongjithesis/。版本之后似乎已停止维护。此版本融合了wildwolf、svandex:https://github.com/svandex/masthesis、zhao-chen:https://github.com/zhao-chen/TongjiThesis~的部分修改。

\section{论文模板使用}

下述给出该模板论文的使用建议。

\subsection{参考文献引用}

由于学校要求在正文中用“等”表示超过3为作者,在参考文献中又要用“et al”来表示超过3为作者,另外中文的“等”前面不需要空格,英文的“等”前需要空格,没找到太好的办法。所以可根据自己需求选择下面的几种效果。

\subsubsection{效果1}

如果有TeX方面的问题可参考资料英文引用\cite{companion},中文引用\cite{shaheshang}。

\subsubsection{效果2}

如果有TeX方面的问题可参考资料英文引用(Goosens等, \citeyear{companion}),中文引用(沙和尚等, \citeyear{shaheshang})。

\subsubsection{效果3}

如果有TeX方面的问题可参考资料Goosens等\citeyearpar{companion},沙和尚等\citeyearpar{shaheshang}。

\subsection{添加公式}

\subsubsection{效果1}

\begin{equation}
	E = m c^2
\end{equation}
其中$E$表示……

\subsection{添加图}

\subsubsection{效果1}

图~\ref{fig:logo}~是我们的校徽和校名,也出现在了本论文模板的封面上。Latex本身有subfigure的命令,但我觉得这个不如用表格来管理图片方便,直观。

\begin{figure}[htb!]
	\centering
	\includegraphics[width=0.5\textwidth]{tongji-whole-logo.eps}
	\caption{校徽和校名}
	\label{fig:logo}
\end{figure}

\subsubsection{效果2}

\begin{figure}[htb!] 
	\centering 
	\begin{tabular}{cc} 
		\includegraphics[width=0.45\textwidth]{tongji-whole-logo.eps} & \includegraphics[width=0.45\textwidth]{tongji-whole-logo.eps}\\ 
		(a)图1 & (b)图2
	\end{tabular} 
	\caption{同济大学} 
	\label{fig:subimage} 
\end{figure}

\subsubsection{效果3}

\begin{figure}[htb!] 
	\noindent 
	\begin{minipage}[t]{.48\linewidth} 
		\centering 
			\includegraphics[width=0.9\textwidth]{tongji-whole-logo.eps} 
		\caption{图1} 
		\label{fig:quadratic} 
	\end{minipage} 
	\begin{minipage}[t]{.48\linewidth} 
		\centering 
		\includegraphics[width=0.9\textwidth]{tongji-whole-logo.eps} 
			\caption{图2} 
		\label{fig:triorder} 
	\end{minipage} 
\end{figure}

\subsection{添加表}

不用在意表~\ref{tab:events}~展示的内容,它只是个示例。

\begin{table}[htb!]
  	\centering
  	\caption{Add caption}
  	\label{tab:events}
    \begin{tabular}{?c"c?}
	    \thickhline
	    a     & b \bigstrut\\
	    \thinhline
		第一列   & 第二列 \bigstrut\\
	    \thinhline
	    第一列   & 第二列 \bigstrut\\
	    \thinhline
	    第一列   & 第二列 \bigstrut\\
	    \thickhline
    \end{tabular}%
  \label{tab:addlabel}
\end{table}

\subsection{其他说明}

未完全测试,尽量使用UTF-8文件编码,xelatex编译,以免出现意外问题。

模板使用了TeXlive默认的Fandol中文字体(Bold支持较好),没有Fandol字体可从:https://www.ctan.org/tex-archive/fonts/fandol~下载;如需使用Windows或者Mac自带的字体,可将cls文件对应的代码取消注释;可如需使用Adobe或其他字体,可自行在cls文件中修改。

\subsection{已知问题}

xelatex编译速度比pdflatex慢,需要耐心等待。

用TeXStudio可能编译时会跳出对话框,提示“日志文件很大,确认载入?”,选择“是”即可照样继续

推荐使用Sublime Text + LaTeXTools,常用快捷键:由Latex查找PDF:ctrl+L(松开接)J;清理临时文件:ctrl+L(松开接)Backspace

\subsection{本章小结}

介绍了同济论文模板的使用。