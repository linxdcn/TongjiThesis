%!TEX root = ../thesis.tex

% 定义中英文摘要和关键字
\begin{cabstract}

近年我国城市轨道交通建设规模持续增长,快速发展模式导致人们忽略了隧道结构长期服役性能,为避免服役性能的劣化导致结构发生不可逆转破坏,有必要对服役性能进行评估、预测性能退化曲线,并提供服役性能分析服务,指导隧道日常养护维护。在服役性能评估方面,单项指标评估方法未能对整体性能作出评判,且不同的单项指标评估得到的结果不同;力学模型的评估方法较难建立考虑真实病害情况的数值模型。在服役性能预测方面,已有的性能退化模型主要考虑时间因素,未考虑数据之间的空间关联性特点。在服役性能服务方面,目前的综合性平台以单体式应用为主,平台庞大,可扩展性弱。因此本文以盾构隧道结构为研究对象,以上海城市轨道交通12号线为工程案例,研究了盾构隧道服役性能定量化的评估方法,建立了考虑空间关联性的服役性能预测模型,以及设计了微服务架构的分析服务。主要工作和研究成果如下:

(1)定义了盾构隧道服役性能(TSI)相关的基本概念和服役性能评估的基本假设,在考虑盾构隧道评估指标获取难度和指标相关性基础上,结合数据之间的相关性,选取六个指标,分别为相对沉降平均值${sett}_{a}$、平均差异沉降$set{{t}_{d\_a}}$、平均收敛变形率${cov}_{a}$、百环渗漏水面积${d}_{l}$、百环衬砌剥落面积${d}_{s}$、百环裂缝长度${d}_{c}$。对39个隧道样本进行专家打分基础上,采用偏最小二乘、主成分分析和典型相关性分析对服役性能TSI回归拟合,结果表明,对早期盾构隧道服役性能(目前隧道样本的运营年限均在20年以内)影响较大的指标依次为相对沉降、收敛变形、渗漏水和差异沉降,衬砌剥落和裂缝两个指标的权重较小,主要原因是早期的剥落并不是运营期间产生的,而是由于施工期的不当操作造成,且在运营期这类病害并没有劣化的趋势。

(2)针对隧道长期服役性能评估(超过20年运营时间),由于缺少实际样本数据,本文提出动态变权函数对长期服役性能评估进行修正,模拟评估指标随着运营时间不断劣化对服役性能的影响,根据对服役性能指标劣化严重性的假设,构造分段状态变权函数,修正TSI公式应用于隧道全寿命周期。

(3)基于影响服役性能因素如周围地层环境、结构上覆荷载等具有空间关联性的假设,采用空间变异理论,将点状、线状的服役性能评估推广为空间网格化评估,宏观上为隧道养护维护工作提供指导。

(4)收集整理隧道历史沉降数据,建立了适用于盾构隧道的自回归滑动平均模型(ARMA)和结构向量(SVAR)时间序列模型。建立的ARMA(3,0)对于沉降二阶差分的拟合$R^2$在0.6以上,原始沉降数据的拟合$R^2$在0.95以上;结构向量模型则引入向量式模拟多维序列的滞后项关联性,和结构式模拟同期项关联性,建立的SVAR(3)模型对于沉降二阶差分的拟合$R^2$在0.75以上,原始沉降数据的拟合$R^2$在0.97以上。考虑空间关联性的模型精准度得到提高,且由SVAR模型也能得出距离更近的监测点关联性更高。

(5)为了改进传统单体式应用模块复杂、体量庞大、可扩展性差等问题,遵循单一功能、独立部署和轻量级通信的原则,设计了盾构隧道服役性能相关的分析服务,包括数据服务、有限元服务和隧道服役性能服务。并研究了微服务的关键技术,如制定不同服务的请求数据和响应数据标准,讨论在不同分析服务功能下的数据交换方式,包括一对一、一对多、同步、异步的通信模式,对于所有分析服务的管理引入服务发现机制,采用注册中心方式对外提供一致性的调用方式。较传统的单体式应用的扩展性更强,对于已有的不同语言开发的分析功能封装高效,为用户提供一种简单获取分析能力的形式。

(6)简单介绍了基础设施智慧服务系统(iS3)的组成,包括基础层、数据层、服务层、应用层和用户层,盾构隧道服役性能微服务成果集成为iS3的服务层,向上对应用层提供服务。另外,本文还开发了iS3 Desktop桌面端和 iS3 Web网页端,在上海地铁12号线工程案例应用中表明,分析服务可为不同应用提供统一的分析能力,辅助管理盾构隧道工程地质勘察、结构设计、运营监测和养护维护各全寿命期阶段。

\end{cabstract}

\ckeywords{服役性能,性能退化,分析微服务,盾构隧道}


\begin{eabstract}

A beard well lathered is half shaved. Proudly writing with \LaTeX{}.

\end{eabstract}

\ekeywords{aaa,bbb,ccc,ddd}